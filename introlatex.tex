%clase de documento
\documentclass[a4paper,12pt]{article}

%paquetes
\usepackage[utf8]{inputenc}
\usepackage[spanish]{babel}
\usepackage[total={18cm, 21cm}, top=2cm, left=2cm]
{geometry}
\usepackage{amsmath, amssymb, amsfonts, latexsym}
\usepackage{graphicx}
\usepackage{color}

%comandos
\parindent = 0mm

\author{Aarón Bravo}
\title{\textsl{LABORATORIO DE FISICA}}
\date{\today}

%contenido
\begin{document}
	\maketitle
	\emph{Algo de texto negro, \color{red}
seguido por un fragmento rojo}, {\color
{blue} finalmente algo de texto azul.}
\textsl{El electromagnetismo es una rama de la física que estudia y unifica los fenómenos eléctricos y magnéticos en \centerline{una sola teoría, cuyos fundamentos fueron presentados }por Michael Faraday y formulados por primera vez de modo {\Huge completo por James Clerk Maxwell.} La formulación consiste en cuatro ecuaciones diferenciales vectoriales \begin{center}<<que relacionan el campo eléctrico, el campo magnético y sus respectivas fuentes materiales corriente eléctrica>> \end{center}, polarización eléctrica y polarización magnética), conocidas como ecuaciones de Maxwell.}
El electromagnetismo es una teoría de campos; es decir, las exp licaciones y\\[-2cm] predicciones que provee se basan en magnitudes físicas vectoriales o tensoriales dependientes de la posición en el espacio y del tiempo. El electromagnetismo describe los fenómenos físicos macroscópicos en los cuales intervienen cargas eléctricas en reposo y en movimiento, usando para ello campos eléctricos y magnéticos y sus efectos sobre las sustancias sólidas, líquidas y gaseosas. Por ser una teoría macroscópica, es decir, aplicable solo a un número muy grande de partículas y a distancias grandes respecto de las dimensiones de estas, el electromagnetismo no describe los fenómenos atómicos y moleculares, para los que es necesario usar la mecánica cuántica.


hola
\end{document}
