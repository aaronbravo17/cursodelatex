%clase de documento
\documentclass[a4paper,20pt]{article}

%paquetes
\usepackage[utf8]{inputenc}
\usepackage[spanish]{babel}
\usepackage[total={18cm, 21cm}, top=2cm, left=2cm]
{geometry}
\usepackage{amsmath, amssymb, amsfonts, latexsym}
\usepackage{graphicx}
\usepackage{color}
\usepackage{multicol}
\usepackage{ mathrsfs }
\usepackage{nicefrac}

%comandos
\parindent = 0mm

\author{Aarón Bravo}
\title{\textsl{LABORATORIO DE FISICA}}
\date{\today}

%contenido
\begin{document}
	\maketitle
	\begin{multicols}{2}
%texto en linea

Sea \(u\in M\) un vector de norma diferente de \(0\), buscamos todos los vectores tales que su producto punto con \(v\) sea \(1\); este no es el ortogonal de \(v\). Es más, dicho conjunto no es un espacio vectorial, es en efecto un espacio afín o un hiperplano. Este hiperplano es de la forma
\[
	H=\{u\in M  : v \cdot u=1\}. . . . . . . . .	\mathscr{A}......\alpha......\beta.......\gamma.........\zeta...........\iota
\]
	\end{multicols}
	
%texto en bloque
\[
	x+yr=0
\],

para todo \(r\in\mathbb{R}\), entonces podemos tomar
\[
	r:=-\dfrac{x}{y}
\]
y [...]

Para todo \( \varepsilon > 0\) existe \(\delta\) > \(0\) tal que.
\[
	|x-a|<\delta \Rightarrow |f(a)-f(x)|< \varepsilon
\]

Este conjunto es rarp:

\[
 \mathbb{R} \times \mathbb{Z} \times \mathbb{Q} / 2.
\]

\begin{equation}
	 \mathbb{R} \times \mathbb{Z} \times \mathbb{Q} / 2.
\end{equation}

\begin{equation}
\dfrac{a}{b} .....\nicefrac{a}{b}....A^2,,,\frac{a}{b}....A_{bi}^{lks}...H^{aloado}
\end{equation}



\( \displaystyle \lim_{x \to a} f(x) = L. \)

Consideremos la sucesión \((x_n)_{n \in \mathbb{N}} \) de termino general

\[
	x_n= \dfrac{(-1)^n}{n+1} .
\]

\[
	\prod_{r \in \mathbb{R}} \int_{\ln 1}^{|r|} \exp (-iz) \sum_{k=1}^{100} \sin (x)\, \cos (kz)\, dz
\]

\[
	\overbrace{\underbrace{(-1)^n}_{}}^{:)}
\]

\[
	Tx=0 \qquad \text{si y sólo si }\qquad x=0..\big\{.\Big[
\]

\[
	\Bigg[ \dfrac{1}{2} \bigg\{4x \left( 8y + 7z \big ( 3x + 4w ( \nicefrac{1}{4} + u)\big ) \big ) \bigg \} \right] \Bigg ] )) \bigg )
\]

\begin{align}

	\sum_{i=0}^{+ \infty} a^i  & = 1 + a + a^2 + a^3 + \ldots &x & = y
	
	\\
		&= \dfrac{1}{1+a} & y & =z 
		
\end{align}

\end{document}